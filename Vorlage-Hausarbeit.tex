% Diese Version nur mit pdfLaTeX benutzen. Für XeLaTeX sind Anpassungen nötig.
% Erste Alpha Version
% 
% Lizenz GPL v3
% You are not allowed to use any of this code for military or similar purposes.
% Copyleft 2014, Niclas Stock
% niclas.stock@me.com
%
% Standing on the shoulders of giants
%
% 

%----------------------------------------------------------------------------------------
%	PRÄAMBEL
%----------------------------------------------------------------------------------------

\RequirePackage[ngerman=ngerman-x-latest]{hyphsubst}
\RequirePackage[l2tabu, orthodox]{nag}
\documentclass[%
	12pt,% hier die Standardschriftgröße wählen
	paper=a4,
	pagesize,
	oneside,
	%DIV=calc,
	%BCOR=1cm,
%----------------------------------------------------------------------------------------%----------------------------------------------------------------------------------------
	draft,% Entfernen vor dem Veröffentlichen
%----------------------------------------------------------------------------------------
%----------------------------------------------------------------------------------------
	ngerman
]{scrreprt}

% Anpassungen und Pakete
%----------------------------------------------------------------------------------------
%	PAKETE
%----------------------------------------------------------------------------------------
\usepackage{fixltx2e} 
\usepackage{eurosym} % Euro Symbol
\usepackage{blindtext}
\usepackage{setspace} % Spacing Einstellungen
\usepackage{titlesec} % Textüberschriften anpassen

%
% pdfLaTeX spezifisch:
%
\usepackage[utf8]{inputenc}
\usepackage[T1]{fontenc}
%\usepackage[sc]{mathpazo} % Palatino Font als serif Alternative zu TNR
\usepackage{mathptmx} % Times New Roman
%\usepackage[scaled=.90]{helvet} % Helvetica als sans serif
\usepackage{inconsolata} % Inconsolas als Monospace
\usepackage[ngerman]{babel} %Deutsche Sonderzeichen und Silbentrennung
\usepackage{textcomp} %diverse Symbole
\DeclareUnicodeCharacter{20AC}{\euro} % Euro Symbol per default Unicode

%
% Mathe etc.
%
\usepackage[decimalsymbol=comma]{siunitx} % SI-Einheiten
%\usepackage{mhchem} % Chemie

%
% development & misc.
%
\usepackage{xspace}
\usepackage[hyphens]{url} %Url Umbruch
\usepackage{remreset} % Fußnoten reset verhindern
%\usepackage{booktabs} %Nette Tabellen
%\usepackage{tcolorbox} % schöne graue Boxen etc.

%
% Typographie
%
\usepackage[babel,german=quotes]{csquotes} % \enquote{Text} für korrekte Anführungszeichen
\usepackage{ellipsis} % Korrigiert den Weißraum um Auslassungspunkte
\usepackage[final]{microtype}

%----------------------------------------------------------------------------------------
%	TYPOGRAPHIEEINSTELLUNGEN%----------------------------------------------------------------------------------------

% Deutsche Einrückweise der Zeilen
\setlength{\parindent}{0em}

% Einrücken der Fußnoten
\deffootnote{1em}{1em}{%
\textsuperscript{\thefootnotemark\ }
}

% Fortlaufende Fußnoten: 
\makeatletter \@removefromreset{footnote}{chapter} \makeatother

%Default Fonts (serif) setzen. insb. fürs Inhaltsverzeichnis
\setkomafont{disposition}{\normalfont} 
\addtokomafont{chapterentry}{\bfseries} 
\addtokomafont{partentry}{\bfseries} 


%Überschriften fürs Inhaltsverzeichnis usw. formatieren
\titleformat{\chapter}[hang]{\normalfont\large\bfseries}{\thechapter\quad}{0pt}{}


%----------------------------------------------------------------------------------------
%	GLIEDERUNGSEBENEN UND TOC
%----------------------------------------------------------------------------------------

% Fügt Gliederungsebene \subA(B+C)paragraph hinzu.
\titleclass{\subAparagraph}{straight}[\subparagraph]
\makeatletter
\newcounter{subAparagraph}[subparagraph]
\newcommand*\l@subAparagraph{\bprot@dottedtocline{6}{14em}{6em}}
\makeatother

\titleclass{\subBparagraph}{straight}[\subAparagraph]
\makeatletter
\newcounter{subBparagraph}[subAparagraph]
\newcommand*\l@subBparagraph{\bprot@dottedtocline{7}{14em}{6em}}
\makeatother

\titleclass{\subCparagraph}{straight}[\subBparagraph]
\makeatletter
\newcounter{subCparagraph}[subBparagraph]
\newcommand*\l@subCparagraph{\bprot@dottedtocline{8}{14em}{6em}}
\makeatother
%Orientierung an den Inhaltsverzeichnissen von C.H. Beck

\makeatletter
\renewcommand{\l@section}{\@dottedtocline{1}{2em}{2.5em}}
\renewcommand{\l@subsection}{\@dottedtocline{2}{4em}{2.5em}}
\renewcommand{\l@subsubsection}{\@dottedtocline{3}{6em}{2.5em}}
\renewcommand{\l@paragraph}{\@dottedtocline{4}{8em}{2.5em}}
\renewcommand{\l@subparagraph}{\@dottedtocline{5}{10em}{2.5em}}
\renewcommand{\l@subAparagraph}{\@dottedtocline{6}{12em}{2.5em}}
\renewcommand{\l@subBparagraph}{\@dottedtocline{7}{14em}{2.5em}}
\renewcommand{\l@subCparagraph}{\@dottedtocline{8}{16em}{2.5em}}
\renewcommand*{\@dotsep}{1}
\makeatother

% Einstellungen für die Gliederungsebenen
% A. I. 1. a) aa) (1) alpha)
\usepackage{alnumsec}
\alnumsectionlevels{0}{chapter,section,subsection,subsubsection,paragraph,subparagraph,subAparagraph,subBparagraph,subCparagraph}
\otherseparators{5}
\surroundarabic[(][)]{}{.}
\surroundgreek[][)]{}{.}
\alnumsecstyle{LRaldagbr}

\renewcommand*{\thepart}{§\,\arabic{part}.}

% Numeriere 6 Ebenen tief
\setcounter{secnumdepth}{8}
% Ebenen im ToC
\setcounter{tocdepth}{8}

%Kopf- und Fußzeilen
\usepackage[automark,headsepline]{scrpage2}
\clearscrplain
\pagestyle{scrplain}
\ohead[\thepage]{\thepage}
\renewcommand{\headfont}{\small\bfseries}

% cleveref (für Verweise)
\usepackage{cleveref}
\crefname{part}{Abschnitt}{Abschnitt}
\crefname{chapter}{Abschnitt}{Abschnitt}
\crefname{section}{Abschnitt}{Abschnitt}
\crefname{subsection}{Abschnitt}{Abschnitt}
\crefname{subsubsection}{Abschnitt}{Abschnitt}
\crefname{paragraph}{Abschnitt}{Abschnitt}
\crefname{subparagraph}{Abschnitt}{Abschnitt}
\crefname{subAparagraph}{Abschnitt}{Abschnitt}
\crefname{subBparagraph}{Abschnitt}{Abschnitt}
\crefname{subCparagraph}{Abschnitt}{Abschnitt}

% Zuverlässiger Reset der Ebenen
%\makeatletter
%\@addtoreset{section}{part} \@addtoreset{subsection}{part} \@addtoreset{subsection}{chapter} \@addtoreset{subsubsection}{part} \@addtoreset{subsubsection}{chapter} \@addtoreset{subsubsection}{section} \@addtoreset{paragraph}{part} \@addtoreset{paragraph}{chapter} \@addtoreset{paragraph}{section} \@addtoreset{paragraph}{subsection} \@addtoreset{subparagraph}{part} \@addtoreset{subparagraph}{chapter} \@addtoreset{subparagraph}{section} \@addtoreset{subparagraph}{subsection} \@addtoreset{subparagraph}{subsubsection} \@addtoreset{subAparagraph}{part} \@addtoreset{subAparagraph}{chapter} \@addtoreset{subAparagraph}{section} \@addtoreset{subAparagraph}{subsection} \@addtoreset{subAparagraph}{subsubsection} \@addtoreset{subAparagraph}{paragraph} \@addtoreset{subBparagraph}{part} \@addtoreset{subBparagraph}{chapter} \@addtoreset{subBparagraph}{section} \@addtoreset{subBparagraph}{subsection} \@addtoreset{subBparagraph}{subsubsection} \@addtoreset{subBparagraph}{paragraph}\@addtoreset{subBparagraph}{subparagraph} \@addtoreset{subCparagraph}{part} \@addtoreset{subCparagraph}{chapter} \@addtoreset{subCparagraph}{section} \@addtoreset{subCparagraph}{subsection} \@addtoreset{subCparagraph}{subsubsection} \@addtoreset{subCparagraph}{paragraph}\@addtoreset{subCparagraph}{subparagraph}\@addtoreset{subCparagraph}{subAparagraph}
%\makeatother

%----------------------------------------------------------------------------------------
%	JURABIB
%----------------------------------------------------------------------------------------
\usepackage{jurabib}
\jurabibsetup{
% Autor kursiv
% Komma zwischen Autor/Bearbeiter und Titel im Zitat; Kurztitel wenn Autor mehr als ein Werk hat
% Alternative "commasep,all" wenn immer Kurztitel ausgegeben werden sollen
titleformat={commasep},
% Bearbeiter kursiv
annotatorformat=italic,
% Bearbeiter nach Bindestrich (Palandt-Grüneberg)
%annotatorlastsep=divis,
% Komma nach Verfasser (vor dem Rest)
commabeforerest,
%Lange Querverweise (auf Festschriften etwa)
crossref={long,dynamic},
% "zitiert als..." im LitVZ, wenn howcited-Feld in BibTeX=1
howcited=normal,
% zitierten Seitenbereich immer ausgeben (always)
pages={test},
bibformat={tabular,ibidem},% Litverz. tabellarisch, mit der-/dieselbe
lookforgender,% Auf das gender-Feld achten, um ders./dies. Zitate zu ermöglichen sm = derselbe, sf = dieselbe, pm oder pf = dieselben (singe male, female; plural male, female)
%superscriptedition=switch,%Hochgestellte Auflage, wenn ssedition=1 in .bib
%dotafter=bibentry,%Punkt nach jedem Eintrag im Lit.verzeichnis
authorformat=dynamic,
}
\jbsuperscripteditionafterauthor
\citetitlefortype{article,periodical,incollection}%Diese immer mit Titel zitieren
\formatpages[~]{article}{(}{)}%Zeitschriften als JZ 2001, 1057, (S.) %[, ]
\formatpages[~]{incollection}{(}{)}%Sammbelbandbeitr"age als FS xy, 1057, (S.) %[, ]

% Bei Festschriften und Zeitschriftenartikeln: "`in"' vor Titel der Sammlung
\renewcommand{\bibjtsep}{in: } 
\renewcommand{\bibbtsep}{in: } 

% Komma statt . vor in:
\renewcommand*{\bibatsep}{,}
\renewcommand*{\bibbdsep}{,}

% Anführungszeichen bei Zeitungsartieln.
\renewcommand*{\ajtsep}{}
\renewcommand*{\bibapifont}[1]{„{#1}“}
\renewcommand*{\jbapifont}[1]{„{#1}“}

% Bei Periodika (AcP et.al.) die Jahreszahl in runde (statt eckige) Klammern setzen.
\renewcommand*{\bibpldelim}{(}
\renewcommand*{\bibprdelim}{)}

% Linke Spalte des Lit.verz. soll ein Drittel der ges. Textbreite einnehmen
\renewcommand*{\bibleftcolumn}{\textwidth/3}

% Nicht Punkt, sondern Komma nach Auflage
\DeclareRobustCommand{\jbaensep}{,}

% Bei Artikeln: Heft-Nummer in Klammern hinter dem Erscheinungsjahr, etwa 2002(7). (aus jurabib-Gruppe)
\DeclareRobustCommand{\artnumberformat}[1]{(#1)}

% Kein Komma hinter Zeitschriftenname (aus: jurabib-Gruppe #661)
\AddTo\bibsgerman{\def\ajtsep{}}
\jbsuperscripteditionafterauthor
\jbsilent % jurabib Warnungen ausschalten (da jblookforgender verwendet wird)
\jbuseidemhrulefalse % derselbe/dieselbe statt ----

% Für URLS
% Zitat mit URL in Fußnote bspw so: \footnote{\cite{pruetting:2011aa}\citefield{url}{pruetting:2011aa} Zugriff am 31.7.2012}
\biburlfont{same}

%----------------------------------------------------------------------------------------
%	// ENDE JURABIB
%----------------------------------------------------------------------------------------


%----------------------------------------------------------------------------------------
%	Ein wenig Commandgefrickel%----------------------------------------------------------------------------------------


% Gegen Schusterjungen und Hurenkinder
% check log! (code von tex.stackexchange.com)
\clubpenalty=152
\widowpenalty=151
% we want to know if we are on first or second column in a 2 column document
\makeatletter
  \def\oncol{\if@twocolumn \space \if@firstcolumn  (first \else (second \fi column)\fi}
\makeatletter
% check if the output penalty was due to orphan or widow or both
\def\testforwidowsandorphans{%
   \ifnum\outputpenalty=151
        \typeout{*** Widow on page  \thepage \oncol}%
  \else
       \ifnum\outputpenalty=152
          \typeout{*** Orphan on page \thepage \oncol}%
      \else
         \ifnum\outputpenalty=303
            \typeout{*** Orphan and Widow on page \thepage \oncol}%
        \fi
      \fi
 \fi
}
% execute this code at the very beginning of the OR
\toks0=\output
\output\expandafter{\expandafter\testforwidowsandorphans
                                   \the\toks0}
%%%%%%%


%Automatisch alle Fußnoten mit einem Punkt abschließen (best feature ever?)
\usepackage{xstring,etoolbox}
\makeatletter
%%% taken from amsthm.sty
\def\@addpunct#1{%
  \relax\ifhmode
    \ifnum\spacefactor>\@m \else#1\fi
  \fi}
\def\nopunct{\spacefactor 1007 }
\def\frenchspacing{\sfcode`\.1006\sfcode`\?1005\sfcode`\!1004%
  \sfcode`\:1003\sfcode`\;1002\sfcode`\,1001 }
%%% end of borrowed code
\patchcmd{\@footnotetext}
  {#1}
  {#1\protect\@addpunct{.}}
  {}{}
\makeatother
% Abkürzungen in separater Datei
% \anf (Anführungszeichen)
%\rcite[Gericht]{Fundstelle}

\newcommand{\anf}[1]{\enquote{#1}}

\usepackage{ifthen}
\newcommand{\rcite}[2][\empty]{%
  \ifthenelse{\equal{#1}{\empty}}
    {#2}
    {\textit{#1} #2}}

%	\zB 		z.B.
%	\zT 		z.T.
%	\mwN 		m.w.N
%	\ua 		u.a.
%	\Abs 		Abs.
%	\Nr 		Nr.
%	\iSd		i.S.d.
%	\iH			i.H.
%	\iHv		i.H.v.
%	\pg 		§
%	\Pg 		§§
%	\rn 		Rn.
%	\Rn			Rn.
%	\Seite 	S.
%	\hM 		h.M. (kursiv)
%	\aA 		a.A. (kursiv)
%	\eA 		e.A. (kursiv)
%	\rspr		Rspr. (kursiv)
%	\Rspr		Rspr. (kursiv)
%	\bgh 		BGH (kursiv)
%	\bverfg	BVerfG (kursiv)

\newcommand{\zB}{z.\,B.\xspace}
\newcommand{\zT}{z.\,T.\xspace}
\newcommand{\mwN}{m.\,w.\,N.\xspace}
\newcommand{\ua}{u.\,a.\xspace}
\newcommand{\Abs}{Abs.\,}
\newcommand{\Nr}{Nr.\,}
\newcommand{\iSd}{i.\,S.\,d.\xspace}
\newcommand{\iH}{i.\,H.\xspace}
\newcommand{\iHv}{i.\,H.\,v.\xspace}
\newcommand{\iVm}{i.\,V.\,m.\xspace}


\newcommand{\pg}{\S\,} 
\newcommand{\Pg}{\SSS\,} 
\newcommand{\rn}{Rn.\,}
\newcommand{\Rn}{Rn.\,} 
\newcommand{\Seite}{S.\,}
\newcommand{\hM}{\emph{h.\,M.}\xspace}
\newcommand{\aA}{\emph{a.\,A.}\xspace}
\newcommand{\eA}{\emph{e.\,A.}\xspace}
\newcommand{\rspr}{\emph{Rspr.}\xspace}
\newcommand{\Rspr}{\emph{Rspr.}\xspace}
\newcommand{\bgh}{\emph{BGH}\xspace}
\newcommand{\BGH}{\emph{BGH}\xspace}
\newcommand{\bverfg}{\emph{BVerfG}\xspace}

\newcommand{\Name}[1]{\emph{#1}}
\newcommand{\Gericht}[1]{\emph{#1}}

%\newcommand{\definition}[1]{\begin{tcolorbox}[boxrule=0.3mm, arc=0mm, title=Definition] #1\end{tcolorbox}}
%\newcommand{\Box}[1]{\begin{tcolorbox}[boxrule=0.3mm, arc=0mm] #1\end{tcolorbox}}


% Draft mode Packages
%%%%%%%%%%%%%%%%%%%%
\usepackage{ifdraft}

\ifoptiondraft{
	\usepackage{color}
	%\usepackage[inner]{showlabels}
	\reversemarginpar \setlength{\marginparwidth}{5.5cm}
	\newcommand{\todo}[1]{\marginpar{\sffamily\scriptsize\textcolor{red}{#1}}}
}

\ifoptionfinal{
\newcommand{\todo}[1]{}
}
%%%%%%%%%%%%%%%%%%%%

% Folgende citekeys ohne Title ausgeben:
% Hier müssen Großkommentare mit mehreren Bänden eingetragen werden, damit diese nicht immer mit dem ganzen Titel zitiert werden (da hier der gleiche shortauthor gesetzt ist (z.B. MüKo) würde ansonsten ein Vollzitat erfolgen)
\nextcitenotitle{%
nk_bgb1:2011,nk_bgb2:2012,mueko_stgb1:2011,mueko_stgb3:2003,lk1:2003,lk1:2006,lk5:2005,mueko_bgb1:2012,mueko_bgb2:2012,mueko_bgb3:2012,mueko_bgb4:2009,mueko_bgb5:2009,mueko_bgb6:2009,staudinger_241-243:2009,staudinger_249-254:2005,staudinger_255-304:2009,staudinger_311-312:2005,staudinger_311b-311c2011,staudinger_315-326:2009,staudinger_328-359:2004,staudinger_433-480:2012,staudinger_433-487:2004,staudinger_311b-311c:2011,staudinger_134-138:2011,staudinger_90-133:2011,palandt:2011,palandt:2012}
%%%%%%%%%%%%%%%%%%%%%%%
\hyphenation{MüKo-BGB}


% Randeinstellungen für das Gutachten
%%%%%%%%%%%%%%%%%%%%%%%%%%%%%%%%%%%%%%%%%%%%%%%%%%%%%%%%%%%%%%%%%%%%%%%%%%%%
%Rand links 7cm, rechts 1cm, oben 1.5cm, unten 1cm, Abstand z. Kopfnote
\usepackage{geometry}
\geometry{lmargin=7cm,rmargin=1cm,tmargin=1.5cm,bmargin=1cm,headsep=3mm}
% Alternativ (genau am Limit (meines) Druckers) :-)
%\geometry{lmargin=7cm,rmargin=0.5cm,tmargin=0.9cm,bmargin=0.6cm,headsep=0cm}

% Option für die Größe der Fußnoten:
%\renewcommand{\footnotesize}{\small}

%----------------------------------------------------------------------------------------
%	DOKUMENT
%----------------------------------------------------------------------------------------
\begin{document}

{
% Randeinstellungen für den Vorspann (InhaltsVZ, etc.)
% left margin 1cm breiter für Steg (je nach Bindung ändern)
\newgeometry{lmargin=3.0cm,rmargin=2cm,tmargin=2.5cm,bmargin=4cm,headsep=1.3cm}

% Römische Ziffern zur Nummerierung des Vorspanns
\pagenumbering{Roman}

%%%%%%%%%%%%%%%
% Titelseite
\begin{titlepage}

%%%%%%%%%%%%%%%%%%%%%%%
%Daten für Titelseite definieren

%Titel der Arbeit eintragen
\title{Verteidigung gegen die dunklen Künste}
%Prof. etc eintragen
\newcommand{\Aufgabensteller}{Prof. S. Snape}
\newcommand{\Semesteranzahl}{2. Fachsemester}
\newcommand{\Semester}{Sommersemester 2012}
% Matrikelnummer eintragen
\author{1234567}

%%%%%%%%%%%%%%%%%%%%%


\begin{flushright}
\makeatletter
% Semester eintragen
\Semesteranzahl \\
% automatisch Datum von Heute 
Matrikel-Nr.~\textbf{\noindent\@author}\\
\hfill\@date
\makeatother
\end{flushright}


\begin{center}
\vspace*{4cm}

{\normalfont
\bfseries\hrulefill

\makeatletter
% Semester eintragen
\Semester\\
\Huge\@title\\
\vspace{0.5cm}
\LARGE\ Hausarbeit
\makeatother

\vspace{1cm}

{\LARGE\Aufgabensteller}

\hrulefill}


\end{center}
\end{titlepage}
 

%%%%%%%%%%%%%%%
% Sachverhalt
\chapter*{Sachverhalt}%Das Sternchen unterdrückt die Nummerierung

% Hier den Sachverhalt reinkopieren.

% Sollte Wert auf die korrekte Wiedergabe (im Bezug auf Satz der Zeichen) gelegt werden, dann sei auf das kommando \\ (also zwei Backslashes) verwiesen, was einen Zweilenumbruch erzwingt.

\blindtext%blindtext
\\ % \\ erwirkt eine leerzeile
\\
\\
\textit{Bearbeitervermerk:} Test 123 456


%Inhaltsverzeichnis (Gliederung) automatisch erstellen
\tableofcontents

%Literaturverzeichnis jurabib.bib einbeziehen
\bibliography{/Users/niclas/Dropbox/JURA/Literatur/juralib.bib} 
\bibliographystyle{jurabib}


% Abkürzungsverzeichnis
\chapter*{Abkürzungsverzeichnis}
\vspace{2cm}

\textit{Hinsichtlich der verwendeten Abkürzungen wird verwiesen auf:}
\\
\\
\textbf{Kirchner, Hildebert/ \\
Butz, Cornelie\\}
Abkürzungsverzeichnis der Rechtssprache. 6. Auflage, Berlin: de Gruyter, 2008
\clearpage


% Abschnitt für die "Gutachten" Seite (also die leerseite auf der nur "Gutachten" steht)
% blöderweise kann auf diese Seite nicht verzichtet werden, da sonst merkwürdigerweise alle Fußnoten verrücken.
%\newgeometry{lmargin=10cm,rmargin=2cm,tmargin=2.5cm,bmargin=4cm,headsep=1.5cm}
\titleformat{\part}[hang]{\huge\raggedleft\bfseries}{\thepart\quad}{0pt}{}
\part*{Gutachten}
}
\restoregeometry % Normale Randeinstellungen


%%%%%%%%%%%%%%
% Hauptteil

% Seitenzahlen mit arabischen Nummern
\pagenumbering{arabic}

% Seitenzahlen "reset"
\setcounter{page}{1}

% Zeilenabstand 1.5
\spacing{1.43}
% spacing{1.43} ~ ca. Word 1,5 Abstand
% spacing erhöht das Default spacing * 1.43, also so wie Word das berechnet (Default (~1,2*1,5). Die .07 die zum *1.5 fehlen sind dem geschuldet, dass LaTeX bei TNR 12pt einen etwas größeren Abstand hat als Word.

% Einzug der Gliederungsebenen {Ebene}{Links der Überschrift}{Über/Links der Überschrift}{Abstand zum Text danach (unter/links)}
\titlespacing{\part}{0em}{1em}{0em}
\titlespacing{\chapter}{0em}{1em}{0em}
\titlespacing{\section}{0em}{0em}{0em}
\titlespacing{\subsection}{0em}{0em}{1em}
\titlespacing{\subsubsection}{0em}{0em}{1em}
\titlespacing{\paragraph}{0em}{0em}{1em}
\titlespacing{\subparagraph}{0em}{0em}{1em}
\titlespacing{\subAparagraph}{0em}{0em}{1em}
\titlespacing{\subBparagraph}{0em}{0em}{1em}
\titlespacing{\subCparagraph}{0em}{0em}{1em}

% Fonts und Formatierung der Gliederungsebenen festlegen
% Als Alternative zum sparen von Zeilen kann man ab \section abwärts statt "hang" "runin" wählen, damit der Text direkt an die (fette) Überschrift anschließt. Um den Abstand zum Text festzulegen muss dann aber bei \titlespacing oben, die letzte Option auf min. 1em (eher mehr) festgelegt werden.
\titleformat{\part}[hang]{\large\centering\bfseries}{\thepart\quad}{0pt}{}
\titleformat{\chapter}[hang]{\normalfont\centering\large\bfseries}{\thechapter\quad}{0pt}{}
\titleformat{\section}[hang]{\normalfont\centering\normalsize\bfseries}{\thesection\quad}{0pt}{}
\titleformat{\subsection}[runin]{\normalfont\normalsize\bfseries}{\thesubsection\quad}{0em}{}
\titleformat{\subsubsection}[runin]{\normalfont\normalsize\bfseries}{\thesubsubsection\quad}{0pt}{}
\titleformat{\paragraph}[runin]{\normalfont\normalsize\bfseries}{\theparagraph\quad}{0pt}{}
\titleformat{\subparagraph}[runin]{\normalfont\normalsize\bfseries}{\thesubparagraph\quad}{0pt}{}
\titleformat{\subAparagraph}[runin]{\normalfont\normalsize\bfseries}{\thesubAparagraph\quad}{0pt}{}
\titleformat{\subBparagraph}[runin]{\normalfont\normalsize\bfseries}{\thesubBparagraph\quad}{0pt}{}
\titleformat{\subCparagraph}[runin]{\normalfont\normalsize\bfseries}{\thesubCparagraph\quad}{0pt}{}
% \part und \chapter schließen direkt an, erzeugen also keine neuen Seiten.
\titleclass{\part}{straight}
\titleclass{\chapter}{straight}



\pagestyle{scrheadings}
\clearscrheadfoot 
\ihead{\headmark} 
\ohead[\thepage]{\thepage}

%----------------------------------------------------------------------------------------
%	Inhalt
%----------------------------------------------------------------------------------------
%%%%%%%%%%%%%%%%%%%%%%%%%%%%%%%%%%%%%%%%%%%%%%%%%%%%%%%%%%%%%%%%%%%
%%%%%%%%%%%%%%%%%%%%%%%%%%%%%%%%%%%%%%%%%%%%%%%%%%%%%%%%%%%%%%%%%%%
% \chapter 			für A.	B.	C.	D.
% \section 			für I. II.	III.IV.
% \subsection			für 1. 2. 	3. 	4.
% \subsubsection		für a)	b)	c)	d)
% \paragraph			für aa) bb)	cc)	dd)
% \subparagraph		für (1) (2)	(3)	(4)
% \subsubparagraph	für α	β	γ 	δ 
%%%%%%%%%%%%%%%%%%%%%%%%%%%%%%%%%%%%%%%%%%%%%%%%%%%%%%%%%%%%%%%%%%%
% \footnote{freier Text, z.B. BGHSt 32,1; \cite[Bearbeiter][Stelle]{CiteKey}
% optional:
% \nopunct am ende der Fußnote wenn kein . gewünscht ist (z.B. bei ! ? \mwN)
%
% \emph{Text} oder textit{Text} = kuriver Text
% \textbf{Text} = fetter Text
% \anf{Text} oder \enquote{Text} = Text in Anführungszeichen
%
% \label{NAME} um Anker zu setzen.
% \cref{NAME} um darauf zu verweisen, bspw: Abschnitt B.II.c)aa)
% \cpageref{NAME} um auf Seite des Ankers zu verweisen.
% 
% optional:
% statt § 123 oder §§ 123, \pg 123 oder \PG 123 benutzen
% statt Rn. 123 \rn 123 benutzen
% statt \emph{BGH} \bgh \hM \aA \eA \rspr \bverfg
% 
% ~ setzt ein Leerzeichen an dem nicht umgebrochen wird.
% Bsp: \pg 123~Abs.~1 StGB
%
% \todo{Hier noch etwas hinschreiben} erzeugt -für die Bearbeitung sehr praktisch- eine farbige Notiz im linken Rand, so dass man schnell seine "Baustellen erkennen kann!
%
% Notfalls: \enlargethispage{\baselineskip} für +1 Zeile
% \looseness=-1 oder 1
%%%%%%%%%%%%%%%%%%%%%%%%%%%%%%%%%%%%%%%%%%%%%%%%%%%%%%%%%%%%%%%%%%%


% Abschnitt "\part" Erster Tatkomplex o.Ä.
% Das sternchen unterdrückt die Nummerierung
\part*{Erster Tatkomplex}
% hierdurch wird es trotzdem ins Inhaltsverzeichnis aufgenommen (ggf. Titel anpassen)
\addcontentsline{toc}{part}{Erster Tatkomplex}
\chapter{Test}\label{test}\thispagestyle{scrplain}
%%%%%%%%%%%%%%%%
% Für die Fußnoten
\renewcommand{\thefootnote}{\fnsymbol{footnote}}\addtocounter{footnote}{1}
\footnotetext{Alle §§ ohne Gesetzesbezeichnung sind solche des BGB; alle Art. solche des GG. Absätze werden durch römische, Sätze durch arabische Zahlen gekennzeichnet.}
\renewcommand{\thefootnote}{\arabic{footnote}}\addtocounter{footnote}{-1}
%%%%%%%%%%%%%%%%








\part*{Zweiter Tatkomplex}
\addcontentsline{toc}{part}{Zweiter Tatkomplex}

\chapter{Dritte Tat}
%%%%%
% das Sternchen unterdrückt die Nummerierung
\chapter*{Endergebnis und Konkurrenzen}
% hierdurch wird es trotzdem ins Inhaltsverzeichnis aufgenommen (ggf. Titel anpassen)
\addcontentsline{toc}{part}{Endergebnis und Konkurrenzen}
% zurücksetzen des Section Counters (damit es hier wieder mit I, II etc. beginnt.)
\setcounter{section}{0}
%%%%%
\section[Kurzname der im InhaltsVZ angezeigt wird]{Erste Tat}

\section{Zweite Tat}

\section{Dritte Tat}


%----------------------------------------------------------------------------------------
%	// Ende Inhalt
%----------------------------------------------------------------------------------------




\pagestyle{scrplain}
%----------------------------------------------------------------------------------------
%	Appendix
%----------------------------------------------------------------------------------------

% Randeinstellungen für den Appendix
\newgeometry{lmargin=3.0cm,rmargin=2cm,tmargin=2.5cm,bmargin=4cm,headsep=1.3cm}
\titlespacing{\chapter}{0pt}{0em}{0pt}
\part*{Versicherung der selbstständigen Anfertigung}
\addcontentsline{toc}{part}{Versicherung der selbstständigen Anfertigung}

% Römische Nummerierung
\pagenumbering{Roman}
%%%%%%%%%%%%%%%%%%%%%%%%%%%%%%%%%%%%%%%%%%%%%%%%%%%%%%%%%%%%%%%%%%%%%%%%%%%%%%%%%%
% den Pagecount (die Römische Zahl oben) muss leider manuell eingestellt werden
\setcounter{page}{12} % hier die Seitenzahl der letzten Seite setzen !!!!!!!!!!
%%%%%%%%%%%%%%%%%%%%%%%%%%%%%%%%%%%%%%%%%%%%%%%%%%%%%%%%%%%%%%%%%%%%%%%%%%%%%%%%%%
\vspace{4em}
Hiermit versichere ich, dass ich die vorliegende Hausarbeit eigenständig und ohne fremde Hilfe angefertigt und keine anderen als die angegebenen Hilfsmittel benutzt habe.
\\

Die Stellen der Hausarbeit, die anderen Quellen im Wortlaut oder dem Sinn nach entnommen wurden, sind durch Angaben der Herkunft kenntlich gemacht.
\vspace{4em}


\makeatletter
Matrikel-Nr.~\textbf{\noindent\@author}
\makeatother

\vspace{3em}

Hannover, den \hspace{1em} \rule{10cm}{0.5pt}

\end{document}



