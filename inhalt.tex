%%%%%%%%%%%%%%%%%%%%%%%%%%%%%%%%%%%%%%%%%%%%%%%%%%%%%%%%%%%%%%%%%%%
%%%%%%%%%%%%%%%%%%%%%%%%%%%%%%%%%%%%%%%%%%%%%%%%%%%%%%%%%%%%%%%%%%%
% \chapter 			für A.	B.	C.	D.
% \section 			für I. II.	III.IV.
% \subsection			für 1. 2. 	3. 	4.
% \subsubsection		für a)	b)	c)	d)
% \paragraph			für aa) bb)	cc)	dd)
% \subparagraph		für (1) (2)	(3)	(4)
% \subsubparagraph	für α	β	γ 	δ 
%%%%%%%%%%%%%%%%%%%%%%%%%%%%%%%%%%%%%%%%%%%%%%%%%%%%%%%%%%%%%%%%%%%
% \footnote{freier Text, z.B. BGHSt 32,1; \cite[Bearbeiter][Stelle]{CiteKey}
% optional:
% \nopunct am ende der Fußnote wenn kein . gewünscht ist (z.B. bei ! ? \mwN)
%
% \emph{Text} oder textit{Text} = kuriver Text
% \textbf{Text} = fetter Text
% \anf{Text} oder \enquote{Text} = Text in Anführungszeichen
%
% \label{NAME} um Anker zu setzen.
% \cref{NAME} um darauf zu verweisen, bspw: Abschnitt B.II.c)aa)
% \cpageref{NAME} um auf Seite des Ankers zu verweisen.
% 
% optional:
% statt § 123 oder §§ 123, \pg 123 oder \PG 123 benutzen
% statt Rn. 123 \rn 123 benutzen
% statt \emph{BGH} \bgh \hM \aA \eA \rspr \bverfg
% 
% ~ setzt ein Leerzeichen an dem nicht umgebrochen wird.
% Bsp: \pg 123~Abs.~1 StGB
%
% \todo{Hier noch etwas hinschreiben} erzeugt -für die Bearbeitung sehr praktisch- eine farbige Notiz im linken Rand, so dass man schnell seine "Baustellen erkennen kann!
%
% Notfalls: \enlargethispage{\baselineskip} für +1 Zeile
% \looseness=-1 oder 1
%%%%%%%%%%%%%%%%%%%%%%%%%%%%%%%%%%%%%%%%%%%%%%%%%%%%%%%%%%%%%%%%%%%


% Abschnitt "\part" Erster Tatkomplex o.Ä.
% Das sternchen unterdrückt die Nummerierung
\part*{Erster Tatkomplex}
% hierdurch wird es trotzdem ins Inhaltsverzeichnis aufgenommen (ggf. Titel anpassen)
\addcontentsline{toc}{part}{Erster Tatkomplex}
\chapter{Test}\label{test}\thispagestyle{scrplain}
%%%%%%%%%%%%%%%%
% Für die Fußnoten
\renewcommand{\thefootnote}{\fnsymbol{footnote}}\addtocounter{footnote}{1}
\footnotetext{Alle §§ ohne Gesetzesbezeichnung sind solche des BGB; alle Art. solche des GG. Absätze werden durch römische, Sätze durch arabische Zahlen gekennzeichnet.}
\renewcommand{\thefootnote}{\arabic{footnote}}\addtocounter{footnote}{-1}
%%%%%%%%%%%%%%%%








\part*{Zweiter Tatkomplex}
\addcontentsline{toc}{part}{Zweiter Tatkomplex}

\chapter{Dritte Tat}
%%%%%
% das Sternchen unterdrückt die Nummerierung
\chapter*{Endergebnis und Konkurrenzen}
% hierdurch wird es trotzdem ins Inhaltsverzeichnis aufgenommen (ggf. Titel anpassen)
\addcontentsline{toc}{part}{Endergebnis und Konkurrenzen}
% zurücksetzen des Section Counters (damit es hier wieder mit I, II etc. beginnt.)
\setcounter{section}{0}
%%%%%
\section[Kurzname der im InhaltsVZ angezeigt wird]{Erste Tat}

\section{Zweite Tat}

\section{Dritte Tat}

