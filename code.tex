%----------------------------------------------------------------------------------------
%	PAKETE
%----------------------------------------------------------------------------------------
\usepackage{fixltx2e} 
\usepackage{eurosym} % Euro Symbol
\usepackage{blindtext}
\usepackage{setspace} % Spacing Einstellungen
\usepackage{titlesec} % Textüberschriften anpassen

%
% pdfLaTeX spezifisch:
%
\usepackage[utf8]{inputenc}
\usepackage[T1]{fontenc}
%\usepackage[sc]{mathpazo} % Palatino Font als serif Alternative zu TNR
\usepackage{mathptmx} % Times New Roman
%\usepackage[scaled=.90]{helvet} % Helvetica als sans serif
\usepackage{inconsolata} % Inconsolas als Monospace
\usepackage[ngerman]{babel} %Deutsche Sonderzeichen und Silbentrennung
\usepackage{textcomp} %diverse Symbole
\DeclareUnicodeCharacter{20AC}{\euro} % Euro Symbol per default Unicode

%
% Mathe etc.
%
\usepackage[decimalsymbol=comma]{siunitx} % SI-Einheiten
%\usepackage{mhchem} % Chemie

%
% development & misc.
%
\usepackage{xspace}
\usepackage[hyphens]{url} %Url Umbruch
\usepackage{remreset} % Fußnoten reset verhindern
%\usepackage{booktabs} %Nette Tabellen
%\usepackage{tcolorbox} % schöne graue Boxen etc.

%
% Typographie
%
\usepackage[babel,german=quotes]{csquotes} % \enquote{Text} für korrekte Anführungszeichen
\usepackage{ellipsis} % Korrigiert den Weißraum um Auslassungspunkte
\usepackage[final]{microtype}

%----------------------------------------------------------------------------------------
%	TYPOGRAPHIEEINSTELLUNGEN%----------------------------------------------------------------------------------------

% Deutsche Einrückweise der Zeilen
\setlength{\parindent}{0em}

% Einrücken der Fußnoten
\deffootnote{1em}{1em}{%
\textsuperscript{\thefootnotemark\ }
}

% Fortlaufende Fußnoten: 
\makeatletter \@removefromreset{footnote}{chapter} \makeatother

%Default Fonts (serif) setzen. insb. fürs Inhaltsverzeichnis
\setkomafont{disposition}{\normalfont} 
\addtokomafont{chapterentry}{\bfseries} 
\addtokomafont{partentry}{\bfseries} 


%Überschriften fürs Inhaltsverzeichnis usw. formatieren
\titleformat{\chapter}[hang]{\normalfont\large\bfseries}{\thechapter\quad}{0pt}{}


%----------------------------------------------------------------------------------------
%	GLIEDERUNGSEBENEN UND TOC
%----------------------------------------------------------------------------------------

% Fügt Gliederungsebene \subA(B+C)paragraph hinzu.
\titleclass{\subAparagraph}{straight}[\subparagraph]
\makeatletter
\newcounter{subAparagraph}[subparagraph]
\newcommand*\l@subAparagraph{\bprot@dottedtocline{6}{14em}{6em}}
\makeatother

\titleclass{\subBparagraph}{straight}[\subAparagraph]
\makeatletter
\newcounter{subBparagraph}[subAparagraph]
\newcommand*\l@subBparagraph{\bprot@dottedtocline{7}{14em}{6em}}
\makeatother

\titleclass{\subCparagraph}{straight}[\subBparagraph]
\makeatletter
\newcounter{subCparagraph}[subBparagraph]
\newcommand*\l@subCparagraph{\bprot@dottedtocline{8}{14em}{6em}}
\makeatother
%Orientierung an den Inhaltsverzeichnissen von C.H. Beck

\makeatletter
\renewcommand{\l@section}{\@dottedtocline{1}{2em}{2.5em}}
\renewcommand{\l@subsection}{\@dottedtocline{2}{4em}{2.5em}}
\renewcommand{\l@subsubsection}{\@dottedtocline{3}{6em}{2.5em}}
\renewcommand{\l@paragraph}{\@dottedtocline{4}{8em}{2.5em}}
\renewcommand{\l@subparagraph}{\@dottedtocline{5}{10em}{2.5em}}
\renewcommand{\l@subAparagraph}{\@dottedtocline{6}{12em}{2.5em}}
\renewcommand{\l@subBparagraph}{\@dottedtocline{7}{14em}{2.5em}}
\renewcommand{\l@subCparagraph}{\@dottedtocline{8}{16em}{2.5em}}
\renewcommand*{\@dotsep}{1}
\makeatother

% Einstellungen für die Gliederungsebenen
% A. I. 1. a) aa) (1) alpha)
\usepackage{alnumsec}
\alnumsectionlevels{0}{chapter,section,subsection,subsubsection,paragraph,subparagraph,subAparagraph,subBparagraph,subCparagraph}
\otherseparators{5}
\surroundarabic[(][)]{}{.}
\surroundgreek[][)]{}{.}
\alnumsecstyle{LRaldagbr}

\renewcommand*{\thepart}{§\,\arabic{part}.}

% Numeriere 6 Ebenen tief
\setcounter{secnumdepth}{8}
% Ebenen im ToC
\setcounter{tocdepth}{8}

%Kopf- und Fußzeilen
\usepackage[automark,headsepline]{scrpage2}
\clearscrplain
\pagestyle{scrplain}
\ohead[\thepage]{\thepage}
\renewcommand{\headfont}{\small\bfseries}

% cleveref (für Verweise)
\usepackage{cleveref}
\crefname{part}{Abschnitt}{Abschnitt}
\crefname{chapter}{Abschnitt}{Abschnitt}
\crefname{section}{Abschnitt}{Abschnitt}
\crefname{subsection}{Abschnitt}{Abschnitt}
\crefname{subsubsection}{Abschnitt}{Abschnitt}
\crefname{paragraph}{Abschnitt}{Abschnitt}
\crefname{subparagraph}{Abschnitt}{Abschnitt}
\crefname{subAparagraph}{Abschnitt}{Abschnitt}
\crefname{subBparagraph}{Abschnitt}{Abschnitt}
\crefname{subCparagraph}{Abschnitt}{Abschnitt}

% Zuverlässiger Reset der Ebenen
%\makeatletter
%\@addtoreset{section}{part} \@addtoreset{subsection}{part} \@addtoreset{subsection}{chapter} \@addtoreset{subsubsection}{part} \@addtoreset{subsubsection}{chapter} \@addtoreset{subsubsection}{section} \@addtoreset{paragraph}{part} \@addtoreset{paragraph}{chapter} \@addtoreset{paragraph}{section} \@addtoreset{paragraph}{subsection} \@addtoreset{subparagraph}{part} \@addtoreset{subparagraph}{chapter} \@addtoreset{subparagraph}{section} \@addtoreset{subparagraph}{subsection} \@addtoreset{subparagraph}{subsubsection} \@addtoreset{subAparagraph}{part} \@addtoreset{subAparagraph}{chapter} \@addtoreset{subAparagraph}{section} \@addtoreset{subAparagraph}{subsection} \@addtoreset{subAparagraph}{subsubsection} \@addtoreset{subAparagraph}{paragraph} \@addtoreset{subBparagraph}{part} \@addtoreset{subBparagraph}{chapter} \@addtoreset{subBparagraph}{section} \@addtoreset{subBparagraph}{subsection} \@addtoreset{subBparagraph}{subsubsection} \@addtoreset{subBparagraph}{paragraph}\@addtoreset{subBparagraph}{subparagraph} \@addtoreset{subCparagraph}{part} \@addtoreset{subCparagraph}{chapter} \@addtoreset{subCparagraph}{section} \@addtoreset{subCparagraph}{subsection} \@addtoreset{subCparagraph}{subsubsection} \@addtoreset{subCparagraph}{paragraph}\@addtoreset{subCparagraph}{subparagraph}\@addtoreset{subCparagraph}{subAparagraph}
%\makeatother

%----------------------------------------------------------------------------------------
%	JURABIB
%----------------------------------------------------------------------------------------
\usepackage{jurabib}
\jurabibsetup{
% Autor kursiv
% Komma zwischen Autor/Bearbeiter und Titel im Zitat; Kurztitel wenn Autor mehr als ein Werk hat
% Alternative "commasep,all" wenn immer Kurztitel ausgegeben werden sollen
titleformat={commasep},
% Bearbeiter kursiv
annotatorformat=italic,
% Bearbeiter nach Bindestrich (Palandt-Grüneberg)
%annotatorlastsep=divis,
% Komma nach Verfasser (vor dem Rest)
commabeforerest,
%Lange Querverweise (auf Festschriften etwa)
crossref={long,dynamic},
% "zitiert als..." im LitVZ, wenn howcited-Feld in BibTeX=1
howcited=normal,
% zitierten Seitenbereich immer ausgeben (always)
pages={test},
bibformat={tabular,ibidem},% Litverz. tabellarisch, mit der-/dieselbe
lookforgender,% Auf das gender-Feld achten, um ders./dies. Zitate zu ermöglichen sm = derselbe, sf = dieselbe, pm oder pf = dieselben (singe male, female; plural male, female)
%superscriptedition=switch,%Hochgestellte Auflage, wenn ssedition=1 in .bib
%dotafter=bibentry,%Punkt nach jedem Eintrag im Lit.verzeichnis
authorformat=dynamic,
}
\jbsuperscripteditionafterauthor
\citetitlefortype{article,periodical,incollection}%Diese immer mit Titel zitieren
\formatpages[~]{article}{(}{)}%Zeitschriften als JZ 2001, 1057, (S.) %[, ]
\formatpages[~]{incollection}{(}{)}%Sammbelbandbeitr"age als FS xy, 1057, (S.) %[, ]

% Bei Festschriften und Zeitschriftenartikeln: "`in"' vor Titel der Sammlung
\renewcommand{\bibjtsep}{in: } 
\renewcommand{\bibbtsep}{in: } 

% Komma statt . vor in:
\renewcommand*{\bibatsep}{,}
\renewcommand*{\bibbdsep}{,}

% Anführungszeichen bei Zeitungsartieln.
\renewcommand*{\ajtsep}{}
\renewcommand*{\bibapifont}[1]{„{#1}“}
\renewcommand*{\jbapifont}[1]{„{#1}“}

% Bei Periodika (AcP et.al.) die Jahreszahl in runde (statt eckige) Klammern setzen.
\renewcommand*{\bibpldelim}{(}
\renewcommand*{\bibprdelim}{)}

% Linke Spalte des Lit.verz. soll ein Drittel der ges. Textbreite einnehmen
\renewcommand*{\bibleftcolumn}{\textwidth/3}

% Nicht Punkt, sondern Komma nach Auflage
\DeclareRobustCommand{\jbaensep}{,}

% Bei Artikeln: Heft-Nummer in Klammern hinter dem Erscheinungsjahr, etwa 2002(7). (aus jurabib-Gruppe)
\DeclareRobustCommand{\artnumberformat}[1]{(#1)}

% Kein Komma hinter Zeitschriftenname (aus: jurabib-Gruppe #661)
\AddTo\bibsgerman{\def\ajtsep{}}
\jbsuperscripteditionafterauthor
\jbsilent % jurabib Warnungen ausschalten (da jblookforgender verwendet wird)
\jbuseidemhrulefalse % derselbe/dieselbe statt ----

% Für URLS
% Zitat mit URL in Fußnote bspw so: \footnote{\cite{pruetting:2011aa}\citefield{url}{pruetting:2011aa} Zugriff am 31.7.2012}
\biburlfont{same}

%----------------------------------------------------------------------------------------
%	// ENDE JURABIB
%----------------------------------------------------------------------------------------


%----------------------------------------------------------------------------------------
%	Ein wenig Commandgefrickel%----------------------------------------------------------------------------------------


% Gegen Schusterjungen und Hurenkinder
% check log! (code von tex.stackexchange.com)
\clubpenalty=152
\widowpenalty=151
% we want to know if we are on first or second column in a 2 column document
\makeatletter
  \def\oncol{\if@twocolumn \space \if@firstcolumn  (first \else (second \fi column)\fi}
\makeatletter
% check if the output penalty was due to orphan or widow or both
\def\testforwidowsandorphans{%
   \ifnum\outputpenalty=151
        \typeout{*** Widow on page  \thepage \oncol}%
  \else
       \ifnum\outputpenalty=152
          \typeout{*** Orphan on page \thepage \oncol}%
      \else
         \ifnum\outputpenalty=303
            \typeout{*** Orphan and Widow on page \thepage \oncol}%
        \fi
      \fi
 \fi
}
% execute this code at the very beginning of the OR
\toks0=\output
\output\expandafter{\expandafter\testforwidowsandorphans
                                   \the\toks0}
%%%%%%%


%Automatisch alle Fußnoten mit einem Punkt abschließen (best feature ever?)
\usepackage{xstring,etoolbox}
\makeatletter
%%% taken from amsthm.sty
\def\@addpunct#1{%
  \relax\ifhmode
    \ifnum\spacefactor>\@m \else#1\fi
  \fi}
\def\nopunct{\spacefactor 1007 }
\def\frenchspacing{\sfcode`\.1006\sfcode`\?1005\sfcode`\!1004%
  \sfcode`\:1003\sfcode`\;1002\sfcode`\,1001 }
%%% end of borrowed code
\patchcmd{\@footnotetext}
  {#1}
  {#1\protect\@addpunct{.}}
  {}{}
\makeatother